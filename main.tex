% !TeX root = main.tex
% !TeX root = ../main.tex
\documentclass[ngerman,10pt]{scrartcl}
\usepackage{microtype} %für bessere Zeilenumbrüche

% Keine Einrückung am Beginn eines Paragraphen
\setlength{\parindent}{0pt}

% Anpassung der Seitenränder
%\usepackage[a4paper,left=1.9cm,right=7.3cm,top=3.0cm,bottom=4.8cm,marginparwidth=5.4cm,marginparsep=4.5mm,heightrounded]{geometry}%right ist der Platz vorgesehen für den margin, marginparwidth ist die Breite des Textes im margin, marginparsep ist der Abstand zum Haupttext, 

\usepackage[a4paper,left=1.9cm,right=7.3cm,top=3.0cm,bottom=4.8cm,marginparwidth=5.4cm,marginparsep=7mm,heightrounded]{geometry}%right ist der Platz vorgesehen für den margin, marginparwidth ist die Breite des Textes im margin, marginparsep ist der Abstand zum Haupttext, 

\setlength\evensidemargin{\oddsidemargin}
%

\usepackage{scrlayer-scrpage}
\usepackage{scrlayer-notecolumn}
% Define the note column
\DeclareNewNoteColumn[
    position=\oddsidemargin+1in+\textwidth+\marginparsep,
    font=\normalfont\footnotesize % Optional font settings for the notes
]{rightnotes}
\newcommand{\note}[1]{\makenote[rightnotes]{#1}}%Marginnotes always on the right side

%Ein \cite liefert keinen hyperlink wenn es als erstes im makenote ausgeführt wird. Daher muss man zuerst mit nocite es zur Bibliography hinzufügen.
\newcommand{\notewithcite}[2]{\nocite{#1}\makenote[rightnotes]{#2}}%Marginnotes always on the right side
%
%
%
% Sprachpacket
\usepackage[english]{babel}

% Die folgenden drei Packete sind für Umlaute wichtig
\usepackage[utf8]{inputenc}
\usepackage{csquotes}

%\renewcommand{\familydefault}{\sfdefault} %sans serif as default font

%Change font of all sectioning titles
\setkomafont{sectioning}{\fontfamily{pplx}\fontseries{b}\selectfont}

%set distance to text after \section{}
%\RedeclareSectionCommands[
%    afterskip=0.3em %
%]{section}

%\setkomafont{chapter}{\LARGE}
%\setkomafont{section}{\large}
%\setkomafont{subsection}{\normalsize}

%Keine neue Zeile nach subsubsection
\RedeclareSectionCommand[
    runin=true,
    afterskip = 2ex plus 0.5ex minus 0.1ex,
]{subsubsection}

%make Section numbering with paragraph sign
%\renewcommand*{\thesection}{\S\arabic{section}}






% !TeX root = ../main.tex


\usepackage{graphicx} % Zum Einbinden von Grafiken
\usepackage{enumerate}
\usepackage[shortlabels]{enumitem}


% generated by the Super Figure vscode extension. May we stand on the shoulder's of giants
\usepackage{import}
\newcommand{\incsvg}[2]{%
    \def\svgwidth{\columnwidth}
    \graphicspath{{#1}}
    \input{#2.pdf_tex}
}

\usepackage{multicol}
\usepackage[fleqn]{amsmath}
\usepackage{amsthm,amssymb}
\usepackage{aligned-overset}
\usepackage{stmaryrd}%für den Widerspruchsblitz
\usepackage{nccmath}
\usepackage{thmtools}
\usepackage{environ}
\usepackage[hidelinks]{hyperref}%dont show red rectangle around references
\usepackage[capitalize,nameinlink]{cleveref}
\usepackage{soul} % for breakable underline \ul



% Define a command to switch to Palatino Bold Italic (only for the theorem titles)
\newcommand{\textpl}[1]{\begingroup\fontfamily{pplx}\selectfont#1\endgroup}
\newcommand{\palantinobolditalic}[1]{\textit{\textbf{\textpl{#1}}}}

%Defining a custom theorem Environment:
\usepackage[most]{tcolorbox}
\usepackage{scrhack}%might be needed for tcolorbox, since tcolorbox uses some KOMA deprecated stuff (see warning when removing \usepackage{scrhack})
\usepackage{environ}
\usepackage{changepage}%for indenting inlinelemma
%save some settings for tcolorbox to reuse later for standard theorem etc
\tcbset{
    mytcbstyle/.style={
        enhanced,
        colback=white,          % Background color of the box
        coltitle=black,         % Color of the theorem title text
        colframe=black,         % Frame color
        boxrule=0.5pt,          % Frame width
        titlerule=0pt,          % No separation rule between title and content
        toptitle=1mm,           % Space at the top inside the box (title)
        bottomtitle=1mm,        % Space at the bottom inside the box (title)
        detach title,           % Do not detach title from the content
        sharp corners,          % Sharp corners
        separator sign none,    % No separator sign
        before skip = 10pt plus 2.5pt,
        after skip = 10pt plus 2.5pt,
    }
}

%save some settings for tcolorbox to reuse later for exercises
\tcbset{
    mytcbstyleforexercises/.style={
        enhanced,
        colback=white,          % Background color of the box
        frame hidden,           %versteckt ramenlinien die übrig bleiben,obwohl man borderlines auf 0pt dicke hat
        coltitle=black,         % Color of the theorem title text
        colbacktitle=white,
        colframe=black,         % Frame color
        boxrule=0pt,          % Frame width
        %borderline north={0.5pt}{0pt}{black}, 
        borderline south={0.5pt}{0pt}{black}, % Horizontal line at the bottom
        attach boxed title to top left={yshift=-1.2mm, xshift=0mm},
        boxed title style={size = minimal, boxrule=0pt,colframe=white,},
        left = 0pt,%padding von inhalt zur linken seite
        boxsep = 0pt,%anderes padding von inhalt zur linken seite
        titlerule=0pt,          % No separation rule between title and content
        toptitle=1mm,           % Space at the top inside the box (title)
        bottomtitle=1mm,        % Space at the bottom inside the box (title)
        detach title,           % Do not detach title from the content
        sharp corners,          % Sharp corners
        separator sign none,    % No separator sign
        before skip = 10pt plus 2.5pt,
        after skip = 10pt plus 2.5pt,
    }
}

%define custom theorem style
\makeatletter
\declaretheoremstyle[
preheadhook=\renewcommand\@upn{},%to make the theoremnumber not always upright see https://tex.stackexchange.com/questions/91487/thmtools-issue-with-head-number-style
numberwithin = section,
spaceabove=6pt, %doesnt work because of tcolorbox?
spacebelow=6pt, %doesnt work because of tcolorbox?
headfont =\fontfamily{pplx}\bfseries\itshape\selectfont,
headformat={\NUMBER\ \NAME: \quad\NOTE\medskip\\},
headpunct =,
notefont=\fontfamily{pplx}\bfseries\itshape\selectfont,%Same as title font, because its in the title
notebraces={}{},%no braces
bodyfont=\normalfont,
postheadspace=1em,
]{mystandardstyle}
\makeatother


%define custom INLINE theorem style
\makeatletter
\declaretheoremstyle[
preheadhook=\renewcommand\@upn{},%to make the theoremnumber not always upright see https://tex.stackexchange.com/questions/91487/thmtools-issue-with-head-number-style
numberwithin = section,
spaceabove=6pt, %doesnt work because of tcolorbox?
spacebelow=6pt, %doesnt work because of tcolorbox?
headfont =\normalfont\itshape\bfseries\selectfont,
headformat={\NAME:},
headpunct =,
notefont=\normalfont\itshape\selectfont,%Same as title font, because its in the title
notebraces={}{},%no braces
bodyfont=\normalfont,
]{myinlinestyle}
\makeatother

%define theorem environments with the above style
\declaretheorem[style=mystandardstyle,name=Definition]{mydefinition}
\declaretheorem[style=mystandardstyle,name=Convention,sibling=mydefinition]{myconvention}
\declaretheorem[style=mystandardstyle,name=Theorem,sibling=mydefinition]{mytheorem}
\declaretheorem[style=mystandardstyle,name=Lemma,sibling=mydefinition]{mylemma}
\declaretheorem[style=mystandardstyle,name=Corollary,sibling=mydefinition]{mycorollary}
\declaretheorem[style=mystandardstyle,name=Remark,sibling=mydefinition]{myremark}
\declaretheorem[style=mystandardstyle,name=Exercise,sibling=mydefinition]{myexercise}
\declaretheorem[style=mystandardstyle,name=Example,sibling=mydefinition]{myexample}
% Declare unnumbered versions
\declaretheorem[style=myinlinestyle,name=Definition,numbered=no]{unnumberedmydefinition}
\declaretheorem[style=myinlinestyle,name=Convention,numbered=no]{unnumberedmyconvention}
\declaretheorem[style=myinlinestyle,name=Theorem,numbered=no]{unnumberedmytheorem}
\declaretheorem[style=myinlinestyle, name=Lemma,numbered=no]{unnumberedmylemma}
\declaretheorem[style=myinlinestyle,name=Corollary,numbered=no]{unnumberedmycorollary}
\declaretheorem[style=myinlinestyle,name=Remark,numbered=no]{unnumberedmyremark}
\declaretheorem[style=myinlinestyle,name=Exercise,numbered=no]{unnumberedmyexercise}
\declaretheorem[style=myinlinestyle,name=Example,sibling=mydefinition]{unnumberedmyexample}
%define the actually used environments with boxes around them
\NewEnviron{definition}[1][]{
    \begin{tcolorbox}[mytcbstyle]
        % Check if the optional argument is provided
        \ifstrempty{#1}{
            % If no argument is provided, omit the brackets
            \begin{mydefinition}
                \BODY
            \end{mydefinition}
        }{
            % If an argument is provided, use it
            \begin{mydefinition}[#1]
                \BODY
            \end{mydefinition}
        }
    \end{tcolorbox}    
}{}
\NewEnviron{convention}[1][]{
    \begin{tcolorbox}[mytcbstyle]
        \ifstrempty{#1}{
            \begin{myconvention}
                \BODY
            \end{myconvention}
        }{
            \begin{myconvention}[#1]
                \BODY
            \end{myconvention}
        }
    \end{tcolorbox}    
}{}

\NewEnviron{theorem}[1][]{
    \begin{tcolorbox}[mytcbstyle]
        % Check if the optional argument is provided
        \ifstrempty{#1}{
            % If no argument is provided, omit the brackets
            \begin{mytheorem}
                \BODY
            \end{mytheorem}
        }{
            % If an argument is provided, use it
            \begin{mytheorem}[#1]
                \BODY
            \end{mytheorem}
        }
    \end{tcolorbox}    
}{}
\NewEnviron{lemma}[1][]{
    \begin{tcolorbox}[mytcbstyle]
        \ifstrempty{#1}{
            \begin{mylemma}
                \BODY
            \end{mylemma}
        }{
            \begin{mylemma}[#1]
                \BODY
            \end{mylemma}
        }
    \end{tcolorbox}    
}{}
\NewEnviron{inlinelemma}[1][]{
    %\begin{adjustwidth}{1cm}{0cm} 
        \begin{addmargin}[0.75cm]{2em}
        \ifstrempty{#1}{
            \begin{unnumberedmylemma}
                \BODY
            \end{unnumberedmylemma}
        }{
            \begin{unnumberedmylemma}[#1]
                \BODY
            \end{unnumberedmylemma}
        }
    \end{addmargin}
    %\end{adjustwidth}
}{}
\NewEnviron{corollary}[1][]{
    \begin{tcolorbox}[mytcbstyle]
        \ifstrempty{#1}{
            \begin{mycorollary}
                \BODY
            \end{mycorollary}
        }{
            \begin{mycorollary}[#1]
                \BODY
            \end{mycorollary}
        }
    \end{tcolorbox}    
}{}
\NewEnviron{remark}[1][]{
        \ifstrempty{#1}{
            \begin{unnumberedmyremark}
                \BODY
            \end{unnumberedmyremark}
        }{
            \begin{unnumberedmyremark}[#1]
                \BODY
            \end{unnumberedmyremark}
        }  
}{}

\NewEnviron{example}[1][]{
        \ifstrempty{#1}{
            \begin{unnumberedmyexample}
                \BODY
            \end{unnumberedmyexample}
        }{
            \begin{unnumberedmyexample}[#1]
                \BODY
            \end{unnumberedmyexample}
        }  
}{}

\NewEnviron{exercise}[1][]{
    \begin{tcolorbox}[mytcbstyleforexercises]
        \ifstrempty{#1}{
            \begin{myexercise}
                \BODY
            \end{myexercise}
        }{
            \begin{myexercise}[#1]
                \BODY
            \end{myexercise}
        }
    \end{tcolorbox}    
}{}



%define theoremstyle for proof environment
\makeatletter
\declaretheoremstyle[
preheadhook=\renewcommand\thmt@space{},%gets rid of space that is added to \NOTE via \thmt@space by redefining the command to be empty
spaceabove=15pt minus 10pt, 
spacebelow=17pt plus 3pt minus 5pt,
headfont =\fontfamily{pplx}\bfseries\itshape\selectfont,
headformat={\NOTE\vspace{2pt}},
headindent=0pt,
postheadhook={\hspace*{\parindent}\newline\thmt@amsthmlistbreakhack},
headpunct =,
notefont=\fontfamily{pplx}\bfseries\itshape\selectfont,%Same as title font, because its in the title
notebraces={}{},%no braces
bodyfont=\normalfont,
qed=\raisebox{-1.5pt}{\scalebox{1.0}{$\blacksquare$}}
]{myproofstyle}
% Amsthm list break hack to reduce space before enumerate
\def\thmt@amsthmlistbreakhack{%
  \leavevmode
  \vspace{-\baselineskip}%
  \par
  \everypar{\setbox\z@\lastbox\everypar{}}%
}
\makeatother
%


%define theoremstyle for inline proof environment
\makeatletter
\declaretheoremstyle[
preheadhook=\renewcommand\thmt@space{},%gets rid of space that is added to \NOTE via \thmt@space by redefining the command to be empty
spaceabove=15pt minus 10pt, 
spacebelow=17pt plus 3pt minus 5pt,
headfont =\normalfont\itshape\selectfont,
headformat={\NOTE\ },
headindent=0pt,
headpunct =,
notefont=\normalfont\itshape\selectfont,%Same as title font, because its in the title
notebraces={}{},%no braces
bodyfont=\normalfont,
qed=\raisebox{-1.5pt}{\scalebox{1.0}{$\blacksquare$}}
]{myinlineproofstyle}
% Amsthm list break hack to reduce space before enumerate
\def\thmt@amsthmlistbreakhack{%
  \leavevmode
  \vspace{-\baselineskip}%
  \par
  \everypar{\setbox\z@\lastbox\everypar{}}%
}
\makeatother
%




%define proof environment
\declaretheorem[style=myproofstyle]{myproof}
\declaretheorem[style=myinlineproofstyle]{myinlineproof}
%overwrite amsthms proof environment
\renewenvironment{proof}[1][Proof]{\begin{myproof}[#1]}{\end{myproof}}
\newenvironment{inlineproof}[1][Proof]{\begin{myinlineproof}[#1]}{\end{myinlineproof}}



\let\oldemph\emph
\renewcommand{\emph}[1]{\oldemph{\ul{#1}}}


\newcommand{\matrixfunc}[2]{
    \begin{bmatrix}
        #1           \\
        \medmath{#2} \\%use package nccmath for \medmath
    \end{bmatrix}}

\newcommand{\twopartdef}[4]
{
    \left\{
    \begin{array}{ll}
        #1 & \text{if } #2 \\
        #3 & \text{if } #4 \\
    \end{array}
    \right.
}
\newcommand{\longtwopartdef}[4]
{
    \left\{
    \begin{array}{ll}
        #1\\
        \quad \text{falls } #2     \\
        #3\\
        \quad \text{sonst } #4     \\
    \end{array}
    \right.
}

\newenvironment{closealign}
{\vspace{1.5ex}\par\noindent\hspace*{\mathindent}$\displaystyle\begin{aligned}}
{\end{aligned}$\vspace{1.5ex}\par}

\newenvironment{indented}
{\vspace{1.5ex}\par\noindent\begin{addmargin}[\mathindent]{0em}}
{\end{addmargin}\vspace{1.5ex}}




%use smaller versions of \land via overwriting the command:
\usepackage{MnSymbol}
\let\MNbot\bot %save the bot symbol from MnSymbol
\DeclareSymbolFont{origsymbols}     {OMS}{cmsy}{m}{n}
\DeclareMathSymbol{\bot}{\mathord}{origsymbols}{"3F} %get the standard bot symbol back
%\DeclareMathSymbol{\setminus}{\mathbin}{origsymbols}{"6E} %get old setminus back.

\let\oldland\land
\renewcommand{\land}{\wedge}
\newcommand{\bigland}{\bigwedge}
\let\oldor\lor
\renewcommand{\lor}{\vee}
\newcommand{\biglor}{\bigvee}
\let\oldimplies\implies
\renewcommand{\implies}{\rightarrow}

\let\oldiff\leftrightarrow
\renewcommand{\iff}{\leftrightarrow}
\newcommand{\false}{\bot}

%\renewcommand{\setminus}{\mathbin{\backslash}}
\renewcommand{\epsilon}{\varepsilon}

\DeclareMathOperator*{\argmin}{argmin}
\DeclareMathOperator*{\argmax}{argmax}

\newcommand{\fa}[1]{\forall#1\:}
\newcommand{\ex}[1]{\exists#1\:}
\newcommand{\exu}[1]{\exists!#1\:}
%Math Macros
\newcommand{\gdw}{\text{\,\,\,gdw.\,\,\,\,}}
\newcommand{\bzw}{\text{bzw.}}
\newcommand{\varlist}{,\kern-0.10em.\kern-0.06em.\kern-0.06em.,} % touching at \kern-0.1725em
\newcommand{\mydots}{\kern-0.10em.\kern-0.06em.\kern-0.06em.\kern-0.1em} % touching at \kern-0.1725em


\newcommand*\diff{\mathop{}d}
\newcommand*\Diff[1]{\mathop{}d^#1}

\renewcommand{\powerset}[1]{\ensuremath{\operatorname{\mathcal{P}}(#1)}}

\newcommand{\quotes}[1]{``#1''}
\input{course_specific.tex}





\title{Mathematics of Reinforcement Learning}

\author{Notes by Moritz Roos}
\date{}
%remove page number from tableofcontents page
%\AtBeginDocument{\addtocontents{toc}{\protect\thispagestyle{empty}}}
\begin{document}
\maketitle

\tableofcontents


\clearpage
\section{Introduction}
\textbf{Goals:}
\begin{itemize}
    \item Look at different types of problems.
    \item Learn the basic principles of Reinforcement Learning (RL).
    \item Lean when to apply RL and when not to?
\end{itemize}

\subsection{It's-a me, Mario!}
We want to fix some terminology by the example of the Super Mario game.
\begin{itemize}
    \item \emph{Agent:} player
    \item \emph{Environment:} game
    \item \emph{state:} the frame/screen being presented to the agent
    \item \emph{Action:} input on the controller
    \item \emph{policy:} a correspondence between states and actions
    \item \emph{episode:} one run of the game from start to finish
    \item \emph{reward:} feedback mechanism telling you how good/bad an action performs in a given state
    \item \emph{return:} criterion to be optimized, typically cumulative (discounted, expected) rewards
    \item \emph{state dynamics:} how the next state is obtained from the current state and the current action
\end{itemize}


\begin{figure}[ht]
    \centering
    %\caption{}
    %\incsvg{path/}{path/file}
    \incsvg{figures}{figures/examplemodell}
    \label{fig:examplemodell}
\end{figure}




\section{Markov decision processes}

\textbf{Goals:}
\begin{itemize}
    \item Develop a mathematical framework for dynamic decision making problems under uncertainty.
    \item Lean how to solve these problems if the model is known.
\end{itemize}

\subsection{Definition of markov decision processes}
We look for stochastic processes (a family of random variables indexed by time) \( \{S_n\}_{n \in  \mathbb{N}_0 }, \{A_n\}_{n \in \mathbb{N}_{0}} \{R_n\}_{n \in \mathbb{N}_{0}} \) modelling the dynamic evolution of states, actions and rewards.

\begin{definition}[Markov decision model]
    A \emph{Markov Decision Model} is a tuple \( (S, A, D, p, r ,\gamma) \) consisting of the following components:
    \begin{enumerate}
        \item a finite set \( S \)  called \emph{state space}.
        \item a finite set \( A \) called \emph{action space}
        \item a set \( D \subseteq S \times A \) whose elements are the \emph{admissible state-action pairs}
        \item a \emph{transition probability function} \( p: S \times S \times A \implies [0,1] \)
              \[
                  (s', s, a) \mapsto p(s' | s, a)\note{This is the probability of ending up in \( s' \) when performing action \( a \) in state \( s \). \( s' \mapsto p(s'| s,a) \) is a probability mass function.}
              \] satisfying
              \[
                  \sum_{s' \in S}p(s' | s, a) = 1
              \] for all \( (s,a) \in S \times A \).
        \item a \emph{reward function} \( r: D \to \mathbb{R} \)
              \[
                  (s,a) \mapsto r(s,a)\note{The reward for performing action \( a \)  in state \( s \).}
              \]
        \item a \emph{discount factor} \( \gamma \in (0,1] \).\note{This encodes the time-value of rewards. \( \gamma^{-1}\cdot r(s,a) \) is the value at time \( 0 \) of receiving \( r(s,a) \) \( n \) steps into the future.}

    \end{enumerate}
\end{definition}


\begin{definition}[Policy \qquad \( \Pi \)  \qquad \( \Pi_d \) ]\label{def_policy}
    A \emph{policy} is a mapping \( \pi: S \times A \to [0,1] \)
    \[
        (s,a) \mapsto \pi(a|s)
    \]
    such that
    \[
        \sum_{a \in A} \pi(a|s) = 1 \note{This means that \( a \mapsto \pi(a|s) \) is a probability mass function for each fixed \( s \) . Thus one action \textit{has} to be chosen.}
    \] for all \( s \in S \) and
    \[
        \pi(a|s) = 0
    \] for all \( s,a \in S\times A \setminus D\).

    We say that \( \pi \) is a \emph{deterministic policy} if
    \[
        \fa{s \in S:}\ex{a \in A:}\pi(a|s) = 1.
    \] We write \( \Pi \) and \( \Pi_d \) for the set of all policies and the subset of deterministic policies respectively.

\end{definition}

To wit, \( \pi(a|s) \) is interpreted as the probability of choosing action \( a \) in state \( s \). One could also introduce \quotes{non-Markovian} policies which depend on the entire history of states and actions or \quotes{non-stationary} policies which depend on the current state and current time.
However, we spare ourselves the trouble, since we will eventually see that there always is a \quotes{stationary, Markovian} optimal policy as in \cref{def_policy}.


\ \\
Intuitively, what happens for a given MDM and a policy \( \Pi \) is the following:

\begin{itemize}
    \item Start in an initial state \( S_{0}:= s_{0} \).
    \item Using the policy \( \pi \), randomly draw an action \( A_{0} \) from \( a \mapsto \pi(a|S_{0}) \).
    \item Collect the reward \( R_{0} := r(S_{0}, A_{0})\) and draw the next state \( S_{1} \) from \( s' \mapsto p(s'|S_{0}, A_{0}) \).
    \item repeat this procedure to construct \( S_{0}, A_{0}, R_{0}, S_{1}, A_{1}, R_{1},\mydots \)
\end{itemize}

Informally, we can then introduce the objective
\[
    \mathbb{E} \left[ \sum_{n=0}^{\infty} \gamma^{n} \cdot R_n \right] = \mathbb{E} \left[ \sum_{n=0}^{\infty} \gamma^{n} \cdot r(S_{n}, A_{n}) \right].
\]


Now we want to construct the mentioned stochastic processes. They should be consistent with our definition of MDM's and policies.\note{This can be seen as the formalization of the intuitive description of what should happen for a given MDM and a policy.} This enables us to make sense of notions like the expected return mentioned above.

For this, the first ingredient is a probability space. We choose the sample space as 
\[
    \Omega := (S \times A)^\infty = \bigtimes_{n=0}^{\infty} (S \times A)
\]
and let the \( \sigma \)-algebra \( \mathcal{A} \) on \( \Omega \) be the power set \( \mathcal{P}(\Omega) \) of \( \Omega \). We can safely do so, since \( \Omega \) is countable. In fact, any element \( w \in \Omega  \) takes the form 
\[
    \omega = (s_{0},a_{0},s_{1},a_{1},s_{2},a_{2},\mydots)
\]  
where \(  \left\{ s_n \right\}_{n \in \mathbb{N}_{0}} \subseteq S, \left\{ a_n \right\}_{n \in \mathbb{N}_{0}} \subseteq A \). With that in mind, we define functions
\[
    S_n(\omega) = S_n((s_{0},a_{0},s_{1},a_{1},\mydots)) := s_n
\]
\[
    A_n(\omega) = A_n((s_{0},a_{0},s_{1},a_{1},\mydots)) := a_n  
\]
for \( n \in \mathbb{N}_{0}, \omega \in \Omega \). 

Then \( S_n, A_n \) trivially\note{Trivially, since we chose the power set as \( \sigma \)-algebra. } are measurable functions ( = random variables) taking values in \( S, A \) respectively.  What is left is showing that there exists a probability measure \( \mathbb{P}^\pi_\mu \) on \( (\Omega, \mathcal{A}) \) such that 
\( S_{0},A_{0}, S_{1},A_{1},\mydots \) have the desired distribution.\note{The \( A \) 's should be determined by the policy and the last state. The \( S \) 's by the transition function and the taken action.}
This can be constructed from \( p, \pi \)  and an initial distribution \( \mu \) on \( S \) such that \( S_{0} \sim \mu \).

What we would like to do is define
\begin{align*}
    \mathbb{P}_\mu^\pi [\left\{ \omega \right\}] &= \mathbb{P}_\mu^\pi[{(s_{0},a_{0},s_{1},a_{1},s_{2},a_{2},\mydots)}] \\
    &:= \mu[\left\{ s_{0} \right\}] \cdot \pi(a_{0} | s_{0}) \cdot p(s_{1} | s_{0},a_{0}) \cdot \pi(a_{1} | s_{1}) \cdot p(s_{2}|s_{1},a_{1}) \cdot \pi(a_{2}|s_{2}) \cdot \mydots \\
    &= \mu[\left\{ s_{0} \right\}] \pi(a_{0},s_{0}) \cdot \prod_{n=1}^\infty p(s_n | s_{n-1}, a_{n-1}) \pi(a_n | s_n) \\
\end{align*}
for all \( \omega = (s_{0},a_{0},s_{1},a_{1},\mydots) \in \Omega \).
The problem is, however, that the resulting probability is usually zero, as this is an infinite product with factors valued in \( \left[ 0,1 \right] \).

\begin{theorem}[Ionescu-Tulcea for MDMs]
    \label{Ionescu-Tulcea}
    Let \( S,A,D,p,r,\gamma \) be an MDM, \( \pi \) a policy, and \( \mu \) a probability measure on \( S \). Then there exists a unique probability measure \( \mathbb{P}_\mu^\pi \) on \( \Omega, \mathcal{A} \) with the property 
    
    \begin{align*}\label{Ionescu-Tulcea-eq}\tag{\textasteriskcentered}
        \quad &\mathbb{P}_\mu^\pi[B \times \bigtimes_{k = n+1}^{\infty} (S \times A)] \\
         &:= \sum_{(s_{0},a_{0},\mydots,s_n,a_n) \in B} \mu[{s_{0}}] \cdot \pi(a_{0}|s_{0}) \cdot \prod_{k=1}^n p(s_k | s_{k-1}, a_{k-1}) \cdot \pi(a_{k} | s_{k})\\
    \end{align*}



    for all \( B \subseteq (S \times A)^{n+1}, n \in \mathbb{N}_0 \). In particular, it holds that 
    \begin{align*}\label{Ionescu-Tulcea-eq-2}\tag{\textasteriskcentered\textasteriskcentered}
        \quad &\mathbb{P}_\mu^\pi [S_{0}=s_{0}, A_{0} = a_{0}, \mydots, S_n = s_n, A_n = a_n] \\
        &= \mu[{s_{0}}] \cdot \pi(a_{0}|s_{0})\cdot \prod_{k=1}^{n} p(s_k | s_{k-1}, a_{k-1}) \cdot \pi(a_k | s_k) \\
    \end{align*} 

    for all \( (s_{0},a_{0},s_{1},a_{1},\mydots,s_n, a_n) \in (S \times A)^{n+1}\) and \( n \in \mathbb{N}_{0} \).  \note{This doesnt yet mean that this \( \mathbb{P}_\mu^\pi \) agrees with \( p \) and \( \pi \). But it will be a consequence later.}
\end{theorem}


\begin{proof}[Proof (idea)]
    The key idea is to take \eqref{Ionescu-Tulcea-eq} as a definition and extend it to the full \( \sigma \)-Algebra. Note that the sets of the form 
    \[
        B \times \bigtimes_{k=n+1}^{\infty} (S \times A) 
    \] 
    are an intersection stable ring of sets generating \( \mathcal{A} = \mathcal{P}(\Omega) \). Moreover, it is not hard to show that \( \mathbb{P}_\mu^\pi \) defined by \eqref{Ionescu-Tulcea-eq} is a well-defined additive set function.
    
    With a bit of work, one can show that \( \mathbb{P}_\mu^\pi \) is even \( \sigma \)-additive, hence it is a pre-measure. The existence and uniqueness of \( \mathbb{P}_\mu^\pi \) on all of \( \mathcal{A} \) is thus a direct consequence of Caratheodory's extension theorem.  
    
    Finally, note that 
    \begin{align*}
       &\left\{ S_{0}=s_{0}, A_{0},a_{0},\mydots,S_n = s_n, A_n = a_n \right\} \\
        &= \left\{ (s_{0},a_{0},a_{1},a_{1},\mydots,s_n,a_n) \right\} \times \bigtimes_{k=n+1}^{\infty}(S \times A),\\
    \end{align*}

    so 
    \begin{align*}
        &\mathbb{P}_\mu^\pi \left[ S_{0}=s_{0}, A_{0},a_{0},\mydots,S_n = a_n  \right] \\
        &\overset{\eqref{Ionescu-Tulcea-eq}}{=} \mu[\left\{ s_{0} \right\}] \pi(a_{0},s_{0}) \prod_{k=1}^{n} p(s_k | s_{k-1}, a_{k-1}) \pi(a_{k}|s_{k}).
    \end{align*}
\end{proof}


Having constructed \( \mathbb{P}_\mu^\pi \), it is now straightforward to show that \( \left\{ S_n \right\}_{n \in \mathbb{N}_{0}}, \left\{ A_n \right\}_{n \in \mathbb{N}_{0}} \) have the desired properties under this measure.

\begin{corollary}[Properties of \( \mathbb{P}_\mu^\pi \) ]
    In the setting of \cref{Ionescu-Tulcea}, it holds that
    \begin{enumerate}
        \item \( \mathbb{P}_\mu^\pi[S_{0} = s] = \mu[\left\{ s \right\}], s \in S \)
        \item \( \mathbb{P}_\mu^\pi [S_{n+1} = s' \,|\, S_n = s, A_n = a] = p(s' \,|\, s,a), n \in \mathbb{N}_{0}, s',s \in S, a \in A \)
        \item  \( \mathbb{P}_\mu^\pi[A_{n} = a \,|\, S_{n} = s] = \pi(a | s) \)
        \item \( \begin{aligned}[t]\note{This is the Markov property of the state action pair evolution.}
             & \mathbb{P}_\mu^\pi [ S_{n+1} = s_{n+1}, A_{n+1} = a_{n+1} \,|\, S_{0} = s_{0}, A_{0} = a_{0}, \mydots, S_n = s_n, A_n = a_n] \\
             &\quad =  \mathbb{P}_{\mu}^\pi[S_{n+1} = s_{n+1}, A_{n+1} = a_{n+1} \,|\, S_n = s_n, A_n = a_n]\\[0.2em]
        \end{aligned} \)   for all \( (s_{0},a_{0},\mydots,s_{n+1}, a_{n+1}) \in (S \times A)^{n+2}, n \in \mathbb{N}_{0} \).
        \item \( \mathbb{P}_\mu^\pi[S_{n+1} = s_{n+1} \,|\, S_{0} = s_{0},\mydots, S_n = s_n] = \mathbb{P}_\mu^\pi[ S_{n+1} = s_n \,|\, S_n = s_n]\) 
        for all \( s_{0},s_{1},\mydots,s_{n+1} \in S^{n+1}, n \in \mathbb{N}_{0}\) if \( \pi \) is deterministic \note{That is, if for each \( s \in S \), there exists \( a \in A \) such that \( \pi(a | s) = 1 \).}.  
    \end{enumerate}
\end{corollary}


\begin{proof}[Proof of (ii)]
    
It holds that
\begin{align*}
    &\mathbb{P}_\mu^\pi [S_{n+1} = s', S_n = s, A_n = a] \\
    &= \sum_{a' \in A} \mathbb{P}_\mu^\pi[ S_{n+1} = s', A_{n+1} = a', S_{n} = s, A_{n} = a]\\
    &\overset{\text{\cref{Ionescu-Tulcea}}}{=} \sum_{a' \in A} \sum_{(s_{0},a_{0}),\mydots,(s_{n-1},\mydots,a_{n-1}) \in S \times A} \bigg( \bigg. \mu[\left\{ s_{0} \right\}] \pi(a_{0}| s_{0}) \prod_{k=1}^{n-1} p(s_k | s_{k-1}, a_{k-1}) \pi(a_{k} | s_{k}) \\
    & \cdot p(s | s_{n-1}, a_{n-1}) \pi(a | s) p(s' | s,a) \pi(a' | s')\bigg.\bigg)\\
    &= p(s' | s, a) \sum_{a' \in A} \pi(a' | s') \cdot \sum_{(s_{0},a_{0}),\mydots,(s_{n-1},a_{n-1})} \mu[\left\{ s_{0} \right\}] \pi(a_{0},s_{0}) \prod_{k=1}^{n-1} p(s_{k} | s_{k-1}, a_{k-1}) \pi(a_{k} | s_{k}) p(s | s_{n-1}, a_{n-1}) \pi(a | s) \\
    &= p(s' | s, a) \sum_{a' \in A} \pi(a' | s) \cdot \mathbb{P}_\mu^\pi [S_n = s, A_n = a] \\
    &= p(s' | s, a) \cdot \mathbb{P}_\mu^\pi [ S_n = s , A_n = a].
\end{align*}

Thus 
\begin{align*}
    \mathbb{P}_\mu^\pi &[ S_{n+1} = s' \,|\, S_n = s , A_n = a]\\
    &= \frac{\mathbb{P}_{\mu}^\pi [S_{n+1} = s, S_n =s ,A_n = a]}{\mathbb{P}_\mu^\pi[ S_n = s, A_n = a]}\\
    &= p(s' | s, a).
\end{align*}

The rest follows by similar arguments.

\end{proof}


Given \( s \in S \), we write \( \delta_s \) for the Dirac measure with mass on \( s \), that is 
\[
    \delta_s[\left\{ s \right\}] = 1, \quad \delta_s[\left\{ s' \right\}] = 0, s' \in S \setminus \left\{ s \right\}
\]
For any \( \pi \in \Pi  \), we then write \( \mathbb{P}_s^\pi := \mathbb{P}_{\delta_s}^\pi\) for the expectation operator with respect to \( \mathbb{P}_\mu^\pi \) and let \( \mathbb{E}_s^\pi = \mathbb{E}_{\delta_s}^\pi \).
 


\subsection{Value Functions and bellman Equations}

\begin{definition}[state value function]
    For any \( \pi \in \Pi \), the function 
    \[ V^\pi: S \to \mathbb{R}, s \mapsto V^\pi(s) := \mathbb{E}_s^\pi[\sum_{n=0}^{\infty}\gamma^n r(S_n, A_n)] \]
    is called the \emph{state value function} (if it exists).
    Moreover, if it exists, the function 
    \[
        V: S \to \mathbb{R}, s \mapsto V(s) := \sup_{\pi \in \Pi} V^\pi (s) = \sup_{\pi \in \Pi} \mathbb{E}_{s}^{\pi}[ \sum_{n=0}^{\infty} \gamma^n r(S_n, A_n)]
    \] 
    is called \emph{(optimal) value function}.

    Finally, a policy \( \pi^\star \in \Pi \) such that \( V^{\pi^\star} := V(s) \) is called \emph{optimal} for the initial state \( s  S \). 

\end{definition}

\begin{remark}
    \( V^\pi \) and \( V \) exist under very mild conditions indeed. Since \( S \) and \( A \) are finite, we conclude that \( |r| \) is bounded. A sufficient condition for the existence of \( V^\pi \) and \( V \) is hence \( \gamma < 1  \) as 
    \[
        \sum_{n=0}^{\infty} \gamma^n | r(S_n, A_n)| \leq \max_{(s,a) \in D} | r(a,s) | \sum_{n=0}^{\infty} \gamma^n = \frac{1}{1-\gamma} \max_{(s,a) \in D} |r(s,a)| \text{a.s.}
    \]        
    for all \( \mathbb{P}_s^\pi \). If the MDM has a finite time horizon, meaning that there is \( N \in \mathbb{N}_{0} \) s.t. \( r(S_n, A_n) = 0 \fa{n \geq N}  \mathbb{P}_s^\pi-\text{a.s.}\) for all \( s \in S,  \in \Pi \), we do not have to assume \( \gamma < 1 \).

\end{remark}

\emph{Standing Assumption:} There exists some \( C > 0 \) s.t. 
\[
    \sum_{n=0}^{\infty} \gamma^n | r(S_n, A_n)| \leq C\quad  \mathbb{P}_s^\pi-\text{a.s.}
\] 
for all \(  \in \Pi, s \in S\). 
While it may seem odd that we are now working with an entire family of optimization problems \( \left\{ V(s) \right\}_{s \in S} \), it is not hard to imagine, that these problems are intimately related and we can learn somethig from that connection.

\begin{lemma}[Time shift]
    \label{time_shift_lemma}
    Let \( s,s' \in S, a \in A \), and \( \pi \in \Pi \).
    Define 
    \[
        \mathbb{P}_{s,\text{a.s.}}^\pi := \mathbb{P}_s^\pi[ \cdot \,|\, S_{0} = s, A_{0} = a, S_{1} = s']
    \] 
    and 
    \[
        \hat{S_n} := S_{n+1}, \hat{A}_n := A_{n+1}, n \in \mathbb{N}_{0}
    \]
    then the distribution of \( \left\{ (\hat{S}_n, \hat{A}_n) \right\}_{n \in \mathbb{N}_{0}} \) under \( \mathbb{P}_{s,\text{a.s.}}^\pi \) is the same as the distribution of 
    \( \left\{ (S_n, A_n) \right\}_{n \in \mathbb{N}_{0}} \) under \( \mathbb{P}_{s'}^\pi \).    
\end{lemma}

\begin{proof}
    For \(  n  \in \mathbb{N}_{0}, (s_{0},a_{0},s_{1},a_{0},\mydots,s_n,a_n) \in (S\times A)^{n+1} \), it suffices to show

    \[
        \mathbb{P}_{s,a,s}^\pi [ \hat{S}_0 = s_{0}, \hat{A}_{0} = a_{0}, \mydots, \hat{S}-n = s_n, \hat{A}_n = a_n] = \mathbb{P}_{s'}^\pi [S_0 = s_{0}, A_{0} = a_{0}, \mydots, S_n = s_n, A_n = a_n].
    \]
    The identity is immediate from the definitions, if \( s_{0} \neq s' \) since both sides are zero in that case. So assume further that \( s_{0} = s' \).  
    Then we have

    \begin{align*}
        &\mathbb{P}_{s,a,s}^\pi[\hat{S}_0 = s', \hat{A}_{0} = a_{0}, \mydots, \hat{S}_n = s_n, \hat{A}_n = a_n] \\
        &= \mathbb{P}_s^\pi[ \hat{S}_{0} = s', \hat{A}_{0} = a_{0}, \mydots, \hat{S}_{0} = s_n, \hat{A}_n = a_n \,|\, S_{0} = s, A_{0} = a, S_{1} = s'] \\
        &= \mathbb{P}_s^\pi [S_{0} =s, A_{0}= a, S_1 = s', A_{1} = a_{0}, \mydots, S_{n+1} = s_n, A_{n+1} = a_n] / \mathbb{P}_s^\pi[S_0 = s, A_{0} = a, s_n = s']\\
        &= \frac{\delta_s[\left\{ s \right\}] \pi(a|s) p(s' | s,a) \pi(a_{0}|s') \prod_{k=1}^n p(s_k | s_{k-1}, a_{k-1}) \pi(a_k | s_k)}{\delta_s[\left\{ s \right\}] \pi(a | s) p(s' | s,a)} \\
        &= \delta_{s'}[\left\{ s' \right\}] \pi(a_{0} | s') \prod_{k=1}^n p(s_k | s_{k-1}, a_{k-1}) \pi(a_k, s_k) = \mathbb{P}_{s'}^\pi[ S_{0} = s', A_{0} = a_{0}, \mydots, S_n = s_n, A_n = a_n]\\ 
    \end{align*}
\end{proof}


\begin{theorem}[Bellman Equation]
    Let \( s \in S, \pi   \). Then 
    \begin{align*}
                V^\pi(s) &= \sum_{s'  S, a \in A} \left[ r(s,a) + \gamma V^\pi(s')  \right] \pi(a | s) p(s' | s,a)\\
                 &= \mathbb{E}_s^\pi \left[ r(S_{0},A_{0}) + \gamma V^\pi(s_{1}) \right] \\
    \end{align*}
\end{theorem}
\begin{proof}
    We have 
    \begin{align*}
        V^\pi(s) &= \mathbb{E}_s^\pi \left[  \sum_{n=0}^{\infty}\gamma^n r(S_n, A_n) \right] \\
        &= \sum_{s' \in S, a\in A} \mathbb{E}_s^\pi \left[ r(S_{0},A_{0}) + \gamma\cdot \sum_{n=1}^\infty \gamma^{n-1} r(S_n, A_n)    \,|\, S_{0} = s, A_{0} = a, S_{1} = s' \right] \cdot \mathbb{P}_s \left[ S_{0} = s, A_{0} = a, S_{1} = s' \right]\\
        &= \sum_{s' \in s, a \in A} \left[ r(s,a) + \gamma \mathbb{E}_s^\pi \left[ \sum_{n=1}^\infty \gamma^{n-1} r(S_n, A_n) | S = s, A_{0} = a, S_{1} = s' \right] \right] \pi(a|s) p(s'| s,a) \\
        &= \sum_{s' \in S, a \in A }\left[ r(s,a) + \gamma \mathbb{E}_{s'}^\pi \left[ \sum_{n=0}^{\infty} \gamma^n r(S_n, A_n) \right] \right] \\
        &= \sum_{s' \in S, a \in A} \left[ r(s,a) + \gamma V^\pi(s') \right] \pi(a| s) p(s' | s,a)
    \end{align*}
    where \( \sum_{n=1}^{\infty} \gamma^{n-1} r(S_n, A_n)  = \sum_{n=0}^{\infty} \gamma^n r(\hat{S}_n, \hat{A}_n) \sim \sum_{n=0}^{\infty} \gamma^n r(S_n, A_n)\) under \( \mathbb{P}_{s'}^\pi \) by \cref{time_shift_lemma}.  
\end{proof}

\end{document}